% Created 2015-11-19 Thu 00:27
\documentclass[a4paper]{tufte-handout}
\usepackage[scaled=0.95]{roboto}
\usepackage{mathpazo}
\linespread{1.05}
\usepackage{eulervm}
\usepackage[usenames]{xcolor}


%% footnote color 
\renewcommand{\thefootnote}{\textcolor{Gray}{\arabic{footnote}}}
\makeatletter

\renewcommand\@footnotetext[2][0pt]{%
  \marginpar{%
    \hbox{}\vspace*{#1}%
    \def\baselinestretch {\setspace@singlespace}%
    \reset@font\footnotesize%
    \@tufte@margin@par% use parindent and parskip settings for marginal text
    \vspace*{-1\baselineskip}\noindent%
    \protected@edef\@currentlabel{%
       \csname p@footnote\endcsname\@thefnmark%
    }%
    \color{Gray}
    \color@begingroup%
       \@makefntext{%
         \ignorespaces#2%
       }%
    \color@endgroup%
  }%
}%

\makeatother
%%==============================================================================
%%                                 SECTIONS
%%==============================================================================
%% section numbering to subsection
\setcounter{secnumdepth}{2}
\renewcommand{\thesection}{\Roman{section}}
\renewcommand{\thesubsection}{\thesection.\Alph{subsection}}
\renewcommand{\thesubsubsection}{\thesubsection\arabic{subsubsection})}
\renewcommand{\theparagraph}{\roman{paragraph}}
 
%%==============================================================================
%%                                 COLORS
%%==============================================================================
%% section format
\titleformat{\section}%
  {\normalfont\Huge\color{Cerulean}}% format applied to label+text
  {\llap{\colorbox{Cerulean}{\parbox{1.5cm}{\hfill\color{white}\thesection}}}}% label
  {1em}% horizontal separation between label and title body
  {}% before the title body
  []% after the title body

% subsection format
\titleformat{\subsection}%
  {\normalfont\Large\itshape\color{TealBlue}}% format applied to label+text
  {\llap{\colorbox{TealBlue}{\parbox{1cm}{\hfill\color{white}\thesubsection}}}}% label
  {0.5em}% horizontal separation between label and title body
  {}% before the title body
  []% after the title body

\renewcommand{\footnotesize}{\scriptsize}
\setcaptionfont{\color{Gray}\footnotesize}
\setsidenotefont{\color{Gray}\footnotesize}
\setmarginnotefont{\color{Gray}\itshape\footnotesize}

\usepackage[utf8]{inputenc}
\usepackage[T1]{fontenc}
\usepackage{graphicx}
\usepackage{longtable}
\usepackage{float}
\usepackage{hyperref}
\usepackage{wrapfig}
\usepackage{rotating}
\usepackage[normalem]{ulem}
\usepackage{amsmath}
\usepackage{textcomp}
\usepackage{marvosym}
\usepackage{wasysym}
\usepackage{amssymb}
\usepackage[scaled=0.9]{zi4}
\usepackage[usenames, dvipsnames]{xcolor}
\usepackage[protrusion=true, expansion=alltext, tracking=true, kerning=true]{microtype}
\usepackage{siunitx}
\usepackage[frenchle, frenchb]{babel}
\usepackage[euler-digits]{eulervm}
\renewcommand{\footnotesize}{\small}
\author{Samuel BARRETO}
\date{\today}
\title{Premières Analyses des Données de Séquençage}
\hypersetup{
 pdfauthor={Samuel BARRETO},
 pdftitle={Premières Analyses des Données de Séquençage},
 pdfkeywords={},
 pdfsubject={},
 pdfcreator={Emacs 24.5.1 (Org mode 8.3.2)}, 
 pdflang={Frenchb}}
\begin{document}

\maketitle

\section{Qualité des données}
\label{sec:orgheadline4}
\subsection{qualité du séquençage}
\label{sec:orgheadline1}
\begin{marginfigure}
  \includegraphics[width=\linewidth]{../untrimmed.png}
  \caption{Qualité des séquences \emph{avant} d'être trimmées et filtrées
      sur la qualité}
\end{marginfigure}

\begin{marginfigure}
  \includegraphics[width=\linewidth]{../trimmed.png}
  \caption{Qualité des séquences \emph{après} avoir été trimmées et filtrées
      sur la qualité}
\end{marginfigure}
Globalement la plupart des séquences était de bonne qualité. Sur les \(192\)
envoyées à séquencer, \(179\) ont été retenues pour l'analyse, soit 93\%.

Étant donnée la faible qualité des bases en début et en fin de séquence, elles
ont été tronquées. Le score \(28\) semblait le seuil naturel de qualité. De plus,
toutes les séquences qui avaient une longueur inférieure à \(620\) étaient
généralement mal alignées. Elles ont été éliminées de l'analyse. 

\subsection{Présence de contaminations ?}
\label{sec:orgheadline2}
\subsection{Cohérence}
\label{sec:orgheadline3}

\newpage
\section{Distribution des SNPs}
\label{sec:orgheadline8}
\subsection{Distribution globale}
\label{sec:orgheadline5}
\begin{figure*}[h]
  \centering
  \includegraphics[width=\linewidth]{../snp_distribution.pdf}
  \caption{La distibution des SNPs, sans tenir compte de la qualité de la
    mutation. La couleur représente le mutant d'origine, qu'il soit sensé être
    Weak ou Strong.}
  \label{figure1}
\end{figure*}

Sans tenir compte de la qualité des mutations, on obtient le résultat en figure
\ref{figure1}. Il semble qu'il n'y ait pas de différence significative dans la
position des mutations entre les deux types de mutants.


Malgré les filtres, il reste du bruit de fond, avec quelques SNP qui ne sont pas
à leur place normale.

Principale conclusion : il y a plus de substitutions dans les régions 3' que 5',
sur la fin de la conversion tract. Où se fait le switch ? 

\marginnote{ À noter qu'on n'a pas de SNP avant la position 61. C'est dû au 
\emph{trimming} des séquences. On perd l'information des premiers SNP. 
}

\subsection{Nombre de SNPs}
\label{sec:orgheadline6}

\begin{center}
\begin{tabular}{llr}
\toprule
 & strong & weak\\
\midrule
nombre de SNP par gène synthétique & 1337 & 1162\\
nombre de substitutions & \textbf{1791} & 708\\
\bottomrule
\end{tabular}
\end{center}

Pour un nombre de SNPs par mutant sensiblement équivalent, il y a \(2.53\) fois
plus de substitutions \emph{strong} que \emph{weak} !

\subsection{Distribution de la qualité des mutation}
\label{sec:orgheadline7}

Le graphe suivant montre un résultat surprenant. Lorsque le gène synthétique est
de type Strong, les substitutions occasionnées sont --- quasiment ---
exclusivement de type \emph{strong}. 

Mais lorsque le mutant est de type Weak, les substitutions occasionnées sont à la
fois de type \emph{weak} et de type \emph{strong}, en fonction de la localisation du SNP.
Les SNP autour de la position 241 et 601 sont tous \emph{strong}. Comment interpréter
ça ?

\begin{figure}[h]
  \centering
  \includegraphics[width=\linewidth]{../substitution_distribution.pdf}
  \caption{\textbf{Distribution des SNP par position sur la séquence de référence.} \\
  On retrouve bien les positions des polymorphismes ``artificiels'', toutes les
  $30$ paires de bases. En vert les mutations \emph{strong} et en rouge les
  mutations \emph{weak}. Les mutants Strong montrent exclusivement des
  substitutions \emph{strong}. Les mutants Weak montrent cependant des
  choses différentes. Il y a beaucoup de mutations \emph{strong}, contrairement
  à l'attendu. 
  }
  \label{figure2}
\end{figure}

\newthought{Montré autrement}, on voit le problème encore plus clairement.  


\begin{figure}[h]
  \centering
  \includegraphics[width=\linewidth]{../muttype_plot.pdf}
  \caption{\textbf{Distribution de la qualité des substitutions}. \\
    À gauche la distribution des substitutions vers GC, à droite celle des
    substitutions vers A ou T. On voit bien que les mutations \emph{weak} sont
    quasiment exclusivement dans les mutants de type Weak, alors qu'on retrouve
    des mutations \emph{strong} dans les deux types de mutants.}
  \label{figure3}
\end{figure}

\clearpage
\section{Distribution de la position de basculement}
\label{sec:orgheadline11}
\subsection{Basculement global}
\label{sec:orgheadline9}
\begin{figure}
  \centering
  \includegraphics[width=\linewidth]{../switch_position_globale.pdf}
  \caption{\textbf{Position des switch, indifféremment de la qualité de la
      substition ou du mutant}. \\
    Il y a des disparités dans la distribution des positions de basculement. Il
    y a beaucoup de basculement dès le début, moins vers la fin. Il semble y
    avoir une sorte de \emph{coldspot} local, autour de $500$bp et $200$bp sur
    la séquence de référence. }
\end{figure}
Il y a des disparités dans la distribution des positions de basculement. Il y a
beaucoup de basculement dès le début, moins vers la fin. Il semble y avoir une
sorte de coldspot local, autour de 500bp sur la séquence de référence. 

\subsection{Position initiale de basculement par type de mutation}
\label{sec:orgheadline10}

\begin{figure}
  \includegraphics[width=\linewidth]{../switch_pos_by_mutant.pdf}
  \caption{Position des switch en fonction du type de mutant. \\
    Le graphe \texttt{A} représente la distribution et la qualité du premier
    SNP, $AT \mapsto GC$ est \emph{strong} et $GC \mapsto AT$ est \emph{weak}.
    Le graphe \texttt{B} représente la distribution du premier SNP par clone, en
    fonction de la qualité du clone, Strong ou Weak. \\
    On ne semble pas voir de différence significative. Dans les deux cas, les
    distributions sont assez similaires pour le \emph{weak} et le \emph{strong}.
    On peut cependant voir des différences entre les graphes \texttt{A} et
    \texttt{B} . Par exemple, toutes les premières substitutions sont de type
    \emph{strong.} \\
    Il y a toujours le même patron de coldspot autour de 541bp.}
\end{figure}

À vu d'œil, il n'y a pas de variation significative sur la distribution des
SNPs, quelle que soit la qualité du gène synthétique ou de la substitution. 

\begin{marginfigure}
  \includegraphics[width=\linewidth]{../end_switch.pdf}
  \caption{Position du dernier SNP.\\
    Pas de variation là dessus. À priori les deux mutants terminent au même
    endroit, c'est à dire au dernier site avant le cutoff de trimming. 
  }
\end{marginfigure}
\end{document}